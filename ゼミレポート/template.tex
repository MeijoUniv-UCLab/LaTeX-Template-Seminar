\documentclass[uplatex]{jsarticle}
\usepackage{UCLabZemiReport}

%%%%%%%%%%%%%%%%%%%%%%%%%%%%%%%%
%          ヘッダ情報          %
%%%%%%%%%%%%%%%%%%%%%%%%%%%%%%%%
\year{YYYY}				% 年度(4桁の西暦)
\MemberID{XXX}			% メンバーID(3桁).連絡網参照
\DocumentNo{NNN}		% ドキュメントNo.(3桁の連番)
\date{YYYY/MM/DD}		% 報告会日(YYYY/MM/DD のフォーマットで)
\author{○○ ○○}		% 報告者氏名(姓名の間に半角スペース)

% 報告会名称(作成するドキュメントに応じて切替)
\MeetingName{第N回ゼミ報告会}	        % 学部生の通常ゼミ
%\MeetingName{第1回大学院ゼミ報告会}	% 修士の通常ゼミ

\makeheader		% ヘッダ・フッタを表示するコマンド(消さない!!)


\begin{document}		% ここから下に報告内容を記述する

%%%%%%%%%%%%%%%%%%%%%%%%%%%%%%%%
%        アブストラクト        %
%%%%%%%%%%%%%%%%%%%%%%%%%%%%%%%%
% アブストラクト:報告内容を完結に書く(行った内容とその結果) 

\begin{abstract}
アブストラクト
\end{abstract}


%%%%%%%%%%%%%%%%%%%%%%%%%%%%%%%%
%             本文             %
%%%%%%%%%%%%%%%%%%%%%%%%%%%%%%%%
% 本文:実施内容,結果,課題,今後の予定などを章節を設定して記述する.

\section{見出し1}

○○


\section{見出し2}

○○


%%%%%%%%%%%%%%%%%%%%%%%%%%%%%%%%
%           参考文献           %
%%%%%%%%%%%%%%%%%%%%%%%%%%%%%%%%
% .texファイルから見た各ファイルの相対パスを設定する
\bibliographystyle{UCLabZemiReport}	% UCLabZemiReport.bst
\bibliography{reference}		    % reference.bib

\end{document}